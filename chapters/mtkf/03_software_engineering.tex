\section{Technical Area: Software Engineering}

To oversee engineering, a manager must understand architectural constraints and the actual lifecycle of the code.

\begin{itemize}
    \item \textbf{Execution Foundation}: Understanding the CAP Theorem (Consistency, Availability, Partition Tolerance). A manager must know that you cannot have all three simultaneously in a distributed system. This underpins decisions on whether a system should be "always available" or "always accurate."
    \item \textbf{Authority Metric}: Evaluation of Cyclomatic Complexity, System Health, and Code Coverage. If the team delivers quickly but test coverage is at 20\%, the manager must be able to understand and evaluate issues of latency, resource consumption, and Disaster Recovery (DR) capability. Furthermore, the manager must recognize that they are accumulating a technical debt "time bomb."
\end{itemize}

\subsection{Technical Competency Map: Software Engineering \& Systems}

\begin{table}[H]
    \centering
    \caption{Software Engineering Competency Map}
    \resizebox{\textwidth}{!}{%
    \begin{tabular}{|l|p{3.5cm}|p{3.5cm}|p{3.5cm}|p{3.5cm}|}
    \hline
    \textbf{Level} & \textbf{Architecture \& Systems Design} & \textbf{Lifecycle \& DevOps (CI/CD)} & \textbf{Proactive Quality \& Observability} & \textbf{Authority Criterion (The "Check")} \\ \hline
    Basic & Distinguishes Monoliths from Microservices and the Client-Server model. & Familiar with Git flows and containers (Docker). Understands minimum infrastructure. & Reads logs and differentiates between application errors (400s) and server/system errors (500s). & Identifies if a team follows engineering standards or is generating technical debt. \\ \hline
    Intermediate & Understands Asynchronous (Queues/Events) vs. Synchronous (REST/gRPC) communication. & Manages CI/CD pipelines and Infrastructure as Code (Terraform/IaC). & Audits test coverage and cyclomatic complexity to prevent systemic failures. & Challenges designs that lack exception handling, retries, or redundancy. \\ \hline
    Advanced & Designs for High Availability. Applies resilience patterns (Circuit Breaker, Saga). & Implements progressive deployments (Blue-Green/Canary) and masters orchestration (K8s). & Establishes SLOs/SLIs. Uses advanced telemetry (Traces/Metrics) to prevent outages. & Decides when to pause the commercial roadmap to prioritize system stability and health. \\ \hline
    Expert & Leads transformations to Event-Driven Architectures and Serverless. & Optimizes Cloud TCO through efficient architecture, not just by billing costs. & Establishes a culture of blameless Post-mortems and continuous resilience improvement. & Acts as the final authority during crises, determining the critical recovery path for the ecosystem. \\ \hline
    \end{tabular}%
    }
\end{table}

\subsection{Knowledge Tree}

\subsubsection{Level 1: Engineering Fundamentals (Basic)}
\begin{itemize}
    \item \textbf{Design Patterns}: Singleton, Factory, Observer, and their impact on maintainability.
    \item \textbf{Asynchronous Programming}: Handling threads, promises, and concurrency for system efficiency.
    \item \textbf{Networking Fundamentals}: Protocols (HTTP/S, TCP/IP), DNS, and basic load balancing.
    \item \textbf{Advanced Versioning}: Branching strategies (Trunk-based development) and Pull Request policies.
\end{itemize}

\subsubsection{Level 2: Architecture and Resilience (Intermediate)}
\begin{itemize}
    \item \textbf{Stability Patterns}: Circuit Breakers, Retries, and Bulkheads to avoid cascading failures.
    \item \textbf{Infrastructure as Code (IaC)}: Concepts of Terraform or CloudFormation to manage "systems" via code.
    \item \textbf{Caching Strategies}: Implementation of Redis/Memcached and invalidation policies.
    \item \textbf{System-to-System Communication}: Service Mesh (Istio), message queues (RabbitMQ/Kafka), and APIs (REST vs. gRPC).
\end{itemize}

\subsubsection{Level 3: Platform Engineering and Operations (Advanced)}
\begin{itemize}
    \item \textbf{Distributed Architectures}: Microservices vs. Modular Monolith and state management.
    \item \textbf{Total Observability}: Implementing the three pillars (Logs, Metrics, Traces) and managing SLIs/SLOs.
    \item \textbf{Systems Security}: DevSecOps, secret management, and vulnerability scanning within the pipeline.
    \item \textbf{CAP Theorem}: Strategic decision-making between Consistency, Availability, and Partition Tolerance.
\end{itemize}
